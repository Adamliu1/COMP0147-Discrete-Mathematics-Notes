\chapter{Euclidean Algorithm}

\section{Euclidean Algorithm Basics}

\begin{definition}[Euclidean Algorithm]
    The \textit{Euclidean Algorithm} can be used to compute the \textit{greatest common divisor} of two integers $a, b \in \Int$, denoted $\gcd(a, b)$.
    
    Its process, given $a \ge b$ is
    \begin{align}
        a           &= q_0 \cdot b + r_1 \\
        b           &= q_1 \cdot r_1 + r_2 \\
        r_1         &= q_2 \cdot r_2 + r_3 \\
                    &\vdots \nonumber \\
        r_{k - 1}   &= q_k \cdot r_k + r_{k + 1} \\
        r_k         &= q_{k + 1} \cdot r_{k + 1} + r_{k + 2} \\
                    &\vdots \nonumber \\
        r_{n - 1}   &= q_n \cdot r_n + r_{n + 1} \\
        r_n         &= q_{n + 1} \cdot r_{n + 1} + 0
    \end{align}
    Such that $\gcd(a, b) \coloneqq r_{n + 1}$.
\end{definition}

\section{gcd(a, b) as a Linear Combination of a and b}

\begin{proposition}
    Given $a, b \in \Int$, then for some $k_1, k_2 \in \Int$, and some $d \in \Int$,
    \begin{equation}
        d = \gcd(a, b) = k_1 a + k_2 b
    \end{equation}
\end{proposition}

\begin{remark}
    To solve the congruence $4 \ast x = 1 \Mod{17}$ for $x$, find $x$ in the form of $x = \Inverse{4} \Mod{17}$.
    
    For instance, to find $\gcd(34, 13)$ as a linear combination $k_1 a + k_2 b$, then first use the Euclidean algorithm to find $\gcd(34, 13)$:
    \begin{equation}
        \left.
        \begin{aligned}
            34 &= 2 \cdot 13 + 8 \\
            13 &= 8 + 5 \\
            8  &= 5 + 3 \\
            5  &= 3 + 2 \\
            3  &= 2 + \boxed{1} \\
            2  &= 2 \cdot 1 + 0
        \end{aligned}
        \quad
        \right\vert
        \quad
        \begin{aligned}
            a   &= 2 \cdot b + r_1 \\
            b   &= r_1 + r_2 \\
            r_1 &= r_2 + r_3 \\
            r_2 &= r_3 + r_4 \\
            r_3 &= r_4 + \boxed{r_5} \\
            r_4 &= 2 \cdot r_5 + 0
        \end{aligned}
    \end{equation}
    Note that
    \begin{equation}
            \begin{aligned}
                a   &= 2 \cdot b + r_1 \\
                b   &= r_1 + r_2 \\
                r_1 &= r_2 + r_3 \\
                r_2 &= r_3 + r_4 \\
                r_3 &= r_4 + \boxed{r_5} \\
                r_4 &= 2 \cdot r_5 + 0
            \end{aligned}
            \quad
            \Leftrightarrow
            \quad
            \begin{aligned}
                r_1         &= a - 2b \\
                r_2         &= b - r_1 \\
                r_3         &= r_1 - r_2 \\
                r_4         &= r_2 - r_3 \\
                \boxed{r_5} &= r_3 - r_4 \\
                \phantom{}  &\phantom{}
            \end{aligned}
        \end{equation}
        It is now possible to \textit{collect} $k_1$ and $k_2$ in a bottom-up manner:
        \begin{align}
            \boxed{r_5} &= r_3 - r_4 \\
                        &= r_3 - (r_2 - r_3) \\
                        &= -r_2 + 2r_3 \\
                        &= -r_2 + 2(r_1 - r_2) \\
                        &= 2r_1 - 3r_2 \\
                        &= 2r_1 - 3(b - r_1) \\
                        &= -3b + 5r_1 \\
                        &= -3b + 5(a - 2b) \\
                        &= 5a - 13b
        \end{align}
        Hence $\gcd(34, 13) = gcd(a, b) = 5a - 13b$ for some $a, b \in \Int$. One may verify this by checking that
        \begin{equation}
            5 \cdot 34 - 13 \cdot 13 = 170 - 169 = 1
        \end{equation}
\end{remark}

\section{Problems for Integers Modulo m}

\begin{itemize}
    \item $\boxed{a \ast x = b \Mod{m} \Leftrightarrow x = \Inverse{a} \ast b \Mod{m}}$
        \subitem For $\PosRat$, given some $a, b, m \in \Int$
        \begin{align}
            a \ast x                                  &= b \Mod{m} \\
            \Leftrightarrow \Inverse{a} \ast a \ast x &= \Inverse{a} \ast b \Mod{m} \\
            \Leftrightarrow x                         &= \Inverse{a} \ast b \Mod{m}
        \end{align}
    \item $\boxed{a^n \Mod{m} \Leftrightarrow (a \cdot a^2 \cdot a^4 \cdot a^8, \dots) \Mod{m}}$
        \subitem That is, to decompose the exponent into smaller equivalences, and use identities such as $a^{\VSBars{G^{\times}_{m}}} = 1 \Mod{m}$.
    \item $\boxed{x^a = b \Mod{m} \Leftrightarrow x = b^{\Inverse{a}} \Mod{m}}$
        \subitem For $\PosRat$, given some $a, b, m \in \Int$
        \begin{align}
            x^a &= b \Mod{m} \\
            x   &= \sqrt[a]{b} \Mod{m} \\
            x   &= b^{\frac{1}{a}} \Mod{m} \\
            x   &= b^{\Inverse{a}} \Mod{m}
        \end{align}
    \item For the discrete logarithm: $\boxed{a^x = b \Mod{m} \Leftrightarrow x = \log_a{b} \Mod{m}}$
\end{itemize}

\section{Multiplicative Group of Integers Modulo m}

\begin{definition}[Relatively Prime, Coprime]
    Two integers $a, b \in \Int$ are \textit{relatively prime} (or \textit{coprime}) if
    \begin{equation}
        \gcd(a, b) = 1
    \end{equation}
\end{definition}

\begin{definition}[Multiplicative Group of mod $m$]
    Given $m \in \Int$, then
    \begin{equation}
        G^{\times}_{m} \coloneqq \set{a \in \Int \mid (1 \le a < m) \land (\gcd(a, b) = 1)}
    \end{equation}
    Forms a group $\left( G^{\times}_{m}, \ast \Mod{m} \right)$ under \textit{multiplicative modulo} $m$.
    
    \begin{enumerate}
        \item \textbf{Closure}
        \begin{equation}
            \Forall a, b, m \in G^{\times}_{m} \colon 
                (\gcd(a, m) = 1) \land (\gcd(b, m) = 1) \to (\gcd(a \ast b, m) = 1)
        \end{equation}
        \item \textbf{Associativity}
            \subitem Given by multiplication on integers modulo $m$.
        \item \textbf{Neutral Element}
        \begin{equation}
            \Forall m \in G^{\times}_{m} \colon \gcd(1, m) = 1
        \end{equation}
        \item \textbf{Invertibility}
        \begin{equation}
            \Forall a \in G^{\times}_{m} \colon \Exists y \in G^{\times}_{m} \colon a \ast y = 1 \Mod{m}
        \end{equation}
        For which the inverse element $y$ is denoted $\Inverse{a}$, giving
        \begin{equation}
            \Forall a \in G^{\times}_{m} \colon a \ast \Inverse{a} = 1 \Mod{m}
        \end{equation}
    \end{enumerate}
\end{definition}

\begin{theorem}[Euler Totient Function]
    Given the \textit{multiplicative modulo group} $G^{\times}_{m}$, then
    \begin{equation}
        \phi(m) \coloneqq \lvert G^{\times}_{m} \rvert
    \end{equation}
\end{theorem}

\begin{theorem}
    If $p$ is prime then
    \begin{equation}
        \phi(p) \equiv p - 1
    \end{equation}
\end{theorem}

\begin{theorem}
    If $p$ is prime and $k \ge 1$ then
    \begin{equation}
        \phi(p^k) \equiv p^{k - 1}(p - 1)
    \end{equation}
\end{theorem}

\begin{theorem}
    If $a, b \in \Int$ and $a, b$ are \textit{relatively prime} (i.e. $\gcd(a, b) = 1$) then
    \begin{equation}
        \phi(ab) \equiv \phi(a) \phi(b)
    \end{equation}
\end{theorem}

\begin{theorem}
    If $a, m \in \Int$ are \textit{relatively prime} (i.e. $\gcd(a, m) = 1$) then
    \begin{equation}
        a^{\phi(m)} = 1 \Mod{m}
    \end{equation}
\end{theorem}

\begin{theorem}[Fermat's Little Theorem]
    Given $p$ is a prime number, then for any $a \in \Int$
    \begin{equation}
        a^p \equiv a \Mod{p}
    \end{equation}
    
    Additionally, if $a, p \in \Int$ are \textit{relatively prime}, $\gcd(a, p) = 1$,
    \begin{equation}
        a^{p - 1} \equiv 1 \Mod{p}
    \end{equation}
\end{theorem}

\begin{remark}
    Given $a \in G^{\times}_{m}$, to find $x$ such that 
    \begin{equation}
        a \ast x = b \Mod{m}
    \end{equation}
    Find $\Inverse{a} \Mod{m}$.
    
    For example, for
    \begin{equation}
        13 \ast x = 6 \Mod{34}
    \end{equation}
    
    Since
    \begin{equation}
        x = \Inverse{13} \ast 6 \Mod{34}
    \end{equation}
    
    Find $\Inverse{13} \Mod{34}$ via the \textit{Euclidean algorithm} which gives
    \begin{equation}
        \Inverse{13} = 21 \Mod{34}
    \end{equation}
    
    Then
    \begin{align}
        x &= 21 \ast 6 \Mod{34} \\
          &= 126 - 3 \ast 34 \Mod{34} \\
          &= 24 \Mod{34}
    \end{align}
\end{remark}

\begin{remark}
    To compute expressions of the form
    \begin{equation}
        a^n \Mod{m}
    \end{equation}
    
    One should decompose $a^n$ to $a^n = a \cdot a^2 \cdot a^4 \cdot \cdots$, and use Fermat's Little Theorem and Euler Totient Function Identities whenever possible.
\end{remark}

\begin{remark}
    For equations of the form
    \begin{equation}
        x^a = b \Mod{m}
    \end{equation}
    Then
    \begin{equation}
        x = b^{\Inverse{a}} \Mod{m}
    \end{equation}
    
    If $\gcd(a, \phi(m)) = 1$ then
    \begin{align}
        a \ast y &= 1 \Mod{\phi(m)} \\
        x        &= b^y \Mod{m}
    \end{align}
    
    If $\gcd(b, m) = 1$, that is if $b, m$ are \textit{relatively prime}
    \begin{align}
        x^a &= (b^y)^a \Mod{m} \\
            &= b^{a \ast y} \Mod{m} \\
            &= b^{1 + k \phi(m)} \Mod{m} \\
            &= b \ast (b^{\phi(m)})^k \Mod{m} \\
            &= b \ast 1^k \Mod{m} \\
            &= b \Mod{m}
    \end{align}
\end{remark}


\section{Rivest–Shamir–Adleman (RSA) Cryptography}

\begin{definition}[RSA, Public Keys and Private Keys]
    Given actors Alice and Bob, the process of RSA is
    \begin{enumerate}
        \item Alice provides \textit{secrete} primes $p$ and $q$.
        \begin{equation}
            n = p \ast q
        \end{equation}
        \item Alice provides two integers $d$ and $e$ such that
        \begin{equation}
            d \ast e = 1 \Mod{\phi(p \ast q)}
        \end{equation}
        \item Alice distributes the pair $(n, e)$ to everyone.
        \item Encryption and Decryption is then
        \begin{align}
            \Encrypt{n}{e}{m} &\coloneqq m^e \Mod{n} \\
            \Decrypt{n}{d}{m} &\coloneqq c^d \Mod{n}
        \end{align}
        \item Bob \textit{encrypts} message $m$ as the encrypted message $c$ where
        \begin{equation}
            c \coloneqq \Encrypt{n}{e}{m}
        \end{equation}
        And sends $c$ to Alice.
        \item Alice \textit{decrypts} $c$ as
        \begin{equation}
            m\prime = \Decrypt{n}{d}{c}
        \end{equation}
        Check that $\gcd(m, n) = 1$, that is if $m, n$ are \textit{relatively prime}, then
        \begin{align}
            m\prime \Mod{n} &= c^d \Mod{n} \\
                            &= (m^e)^d \Mod{n} \\
                            &= m^{d \ast e} \Mod{n} \\
                            &= m^{1 + k\phi(p \ast q)} \Mod{n} \\
                            &= m \Mod{n}
        \end{align}
        Then \textit{only} Alice can decrypt the encrypted message $c$ in polynomial time.
    \end{enumerate}
\end{definition}

\begin{remark}
    An example of the RSA process:
    \begin{enumerate}
        \item Alice provides secret primes $p = 3, q = 41$
        \begin{equation}
            n = 3 \ast 41 = 123
        \end{equation}
        \item Alice provides two integers $d = 27, e = 3$
        \begin{align}
            d \ast e \Mod{\phi(3 \ast 41)}  &= 27 \ast 3 \Mod{\phi(3 \ast 41)} \\
                                            &= 81 \Mod{[\phi(3) \ast \phi(41)]} \\
                                            &= 81 \Mod{[2 \ast 40]} \\
                                            &= 81 \Mod{80} \\
                                            &= 1 \Mod{80}
        \end{align}
        \item Alice distributes $(n, e) = (123, 3)$ to everyone.
        \item The encryption and decryption functions are
        \begin{align}
            \Encrypt{n}{e}{m} &= m^3 \Mod{n} \\
            \Decrypt{n}{d}{c} &= c^{27} \Mod{n}
        \end{align}
        \item Given a message $m = 5$ then Bob sends
        \begin{align}
            c &= 5^3 \Mod{123} \\
              &= 125 \Mod{123} \\
              &= 2 \Mod{123}
        \end{align}
        \item Alice receives the encrypted message $c = 2$ and decrypts with the fact that  $\gcd(123, 5) = 1$
        \begin{align}
            m\prime \Mod{123} &= 2^{27} \Mod{123} \\
                              &= 5 \Mod{123}
        \end{align}
    \end{enumerate}
\end{remark}
