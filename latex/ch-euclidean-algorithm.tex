\chapter{Euclidean Algorithm}

\section{Euclidean Algorithm Basics}

\begin{definition}[Euclidean Algorithm]
    The \textit{Euclidean Algorithm} can be used to compute the \textit{greatest common divisor} of two integers $a, b \in \Int$, denoted $\gcd(a, b)$.
    
    Its process, given $a >= b$ is
    \begin{align}
        a           &= q_0 \cdot b + r_1 \\
        b           &= q_1 \cdot r_1 + r_2 \\
        r_1         &= q_2 \cdot r_2 + r_3 \\
                    &\vdots \nonumber \\
        r_{k - 1}   &= q_k \cdot r_k + r_{k + 1} \\
        r_k         &= q_{k + 1} \cdot r_{k + 1} + r_{k + 2} \\
                    &\vdots \nonumber \\
        r_{n - 1}   &= q_n \cdot r_n + r_{n + 1} \\
        r_n         &= q_{n + 1} \cdot r_{n + 1} + 0
    \end{align}
    Such that $\gcd(a, b) \coloneqq r_{n + 1}$.
\end{definition}

\section{gcd(a, b) as a Linear Combination of a and b}

\begin{proposition}
    For some integers $k_1$ and $k_2$, and some $d \in \Int$,
    \begin{equation}
        d = \gcd(a, b) = k_1 a + k_2 b
    \end{equation}
\end{proposition}

\begin{remark}
    For example, to find $\Inverse{4} \pmod{17}$, that is to find $y$ such that $4 \ast y = 1 \pmod{17}$.
\end{remark}