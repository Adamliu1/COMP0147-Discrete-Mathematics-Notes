\chapter{Groups}

\section{Group Basics}

A \textit{group} is an abstract collection consisting of:
\begin{itemize}
    \item A \textit{nonempty set} $G$.
    \item A \textit{binary operation} $\star \colon G \times G \to G$.
\end{itemize}

It has the following properties:
\begin{enumerate}
    \item \textbf{Closure}
    \begin{equation}
        \Forall x, y \colon x \in G \land y \in G \to x \star y \in G
    \end{equation}
    \item \textbf{Associativity}
    \begin{equation}
        \Forall x, y, z \in G \colon (x \star y) \star z \equiv x \star (y \star z)
    \end{equation}
    \item \textbf{Neutral Element}
    \begin{equation}
        \Exists \epsilon \in G \colon \Forall x \in G \colon x \star \epsilon \equiv \epsilon \star x \equiv x
    \end{equation}
    That there exists an unique \textit{neutral} element $\epsilon \in G$.
    \item \textbf{Invertibility}
    \begin{equation}
        \Forall x \in G \colon \Exists y \in G \colon x \star y \equiv y \star x \equiv \epsilon
    \end{equation}
    That there exists an unique \textit{inverse} element $y \coloneqq \Inverse{x} \in G$ where $\Inverse{x}$ denotes the \textit{inverse} element of $x$.
\end{enumerate}

\begin{definition}[Commutative Group]
    An \textit{commutative group} (or \textit{abelian group}) is a \textit{group} for which its operation $\star \colon G \times G \to G$ satisfies the additional \textit{commutative} property:
    \begin{itemize}
        \item \textbf{Commutativity}
        \begin{equation}
            \Forall x, y \in G \colon x \star y \equiv y \star x
        \end{equation}
    \end{itemize}
\end{definition}

\section{Multiplicative Group}

\begin{proposition}[Multiplicative Group]
    A \textit{multiplicative group} is a \textit{group} $(G, \ast)$ which has the binary operation $\ast \colon G \times G \to G$:
    \begin{itemize}
        \item \textbf{Closure, Associativity}. The multiplication operation $\ast \colon G \times G \to G$ is closed and is left associative.
        \item \textbf{Neutral Element}. The neutral element $\epsilon$ is unique.
        \item \textbf{Invertibility}. The inverse element $\Inverse{x}$ is unique.
        \item For all $a, b \in G$ the equation
        \begin{equation}
            a \ast x = b
        \end{equation}
        Has the unique solution
        \begin{equation}
            x = \Inverse{a} \ast b
        \end{equation}
    \end{itemize}
    Since
    \begin{align}
        a \ast x = b &\Leftrightarrow \Inverse{a} \ast (a \ast x) = \Inverse{a} \ast b &\text{(Multiply by inverse element)} \\
        &\Leftrightarrow (\Inverse{a} \ast a) \ast x = \Inverse{a} \ast b &\text{(Associativity)} \\
        &\Leftrightarrow \epsilon \ast x = \Inverse{a} \ast b &\text{(Invertibility)} \\
        &\Leftrightarrow x = \Inverse{a} \ast b &\text{(Neutral Element)}
    \end{align}
\end{proposition}

\begin{remark}
    An example of a multiplicative group is permutations under composition, namely $S_n$ is a group $(G, \circ)$ where $\circ \colon G \times G \to G$.
    
    For example, let $G$ be the set of permutations
    \begin{equation}
    \epsilon = \begin{pmatrix}
        1 & 2 & 3 \\
        1 & 2 & 3
    \end{pmatrix} \quad
    \sigma_1 = \begin{pmatrix}
        1 & 2 & 3 \\
        2 & 3 & 1
    \end{pmatrix} \quad
    \sigma_2 = \sigma_1^2 = \begin{pmatrix}
        1 & 2 & 3 \\
        3 & 1 & 2
    \end{pmatrix} \quad
    \end{equation}
    
    To verify that $G$ does form a group with composition $\circ$, one may draw the multiplication table for the group. Note that
    \begin{equation}
        \sigma_2 \sigma_2 = \sigma_1^4 = \sigma_1^3 \sigma_1 = \epsilon \sigma_1 = \sigma_1
    \end{equation}
    
    \begin{table}[H]
    \centering
    \begin{tabular}{l | l l l}
    \toprule
    $\circ$    & $\epsilon$ & $\sigma_1$ & $\sigma_2$ \\
    \midrule
    $\epsilon$ & $\epsilon$ & $\sigma_1$ & $\sigma_2$ \\
    $\sigma_1$ & $\sigma_1$ & $\sigma_2$ & $\epsilon$ \\
    $\sigma_2$ & $\sigma_2$ & $\epsilon$ & $\sigma_1$ \\
    \bottomrule
    \end{tabular}
    \caption{Multiplication Table of Composition $\circ$ over $G$}
    \end{table}
\end{remark}

\section{Additive Group}

\begin{definition}[Additive Group]
    An \textit{additive group} is a \textit{group} $(G, +)$ with the binary operation $+ \colon G \times G \to G$. It has the same properties of a general \textit{group}.
    \begin{enumerate}
        \item \textbf{Closure}
        \begin{equation}
            \Forall x, y \colon x \in G \land y \in G \to x + y \in G
        \end{equation}
        \item \textbf{Associativity}
        \begin{equation}
            \Forall x, y, z \in G \colon (x + y) + z \equiv x + (y + z)
        \end{equation}
        \item \textbf{Neutral Element}
        \begin{equation}
            \Exists \epsilon \in G \colon \Forall x \in G \colon x + \epsilon \equiv \epsilon + x \equiv x
        \end{equation}
        That there exists an unique \textit{neutral} element $0_G \in G$ (usually denoted simply as $0$).
        \item \textbf{Invertibility}
        \begin{equation}
            \Forall x \in G \colon \Exists y \in G \colon x + y \equiv y + x \equiv 0
        \end{equation}
        That there exists an unique \textit{inverse} element $y \coloneqq -x \in G$ where $-x$ denotes the \textit{inverse} element of $x$.
    \end{enumerate}
\end{definition}

\begin{remark}
    An example of an additive group is $(\Int, +)$ (i.e. addition over the integers).
    
    Then for any of such \textit{commutative group} $(G, +)$
    \begin{itemize}
        \item \textit{Neutral element} $0$ is unique.
        \item \textit{Inverse element} $-x$ is unique.
        \item For any $a, b \in G$ the equation
        \begin{equation}
            a + x = b
        \end{equation}
        Has a unique solution
        \begin{equation}
            x = b + (-a) = b - a
        \end{equation}
    \end{itemize}
\end{remark}

\section{Associativity of Sequential Composition of Functions}

\begin{definition}[Sequential Composition of Functions]
    Let $f \ast g$ denote the sequential composition of functions $f \ast X \to Y$ and $g \colon Y \to Z$ such that $f \ast g \colon X \to Z$ where $f$ is applied first then $g$, i.e. $\Forall x \in X \colon (f \ast g)(x) \coloneqq g(f(x))$.
\end{definition}

\begin{proposition}[Associativity of Sequential Composition of Functions]
    Given sets $X, Y$ and $Z$ and
    \begin{itemize}
        \item \textit{Injection} $f \colon A \to B$
        \item \textit{Injection} $g \colon B \to C$
        \item \textit{Injection} $h \colon C \to D$
    \end{itemize}
    
    Then their composition is associative:
    \begin{equation}
        (f \ast g) \ast h \equiv f \ast (g \ast h)
    \end{equation}
\end{proposition}

\begin{proof}\ \\
    Let $s = (f \ast g)$ and $t = (s \ast h)$, then $t(x) = h(s(x)) = h(g(f(x)))$.\\
    Let $u = (g \ast h)$ and $v = (f \ast u)$, then $v(x) = u(f(x)) = h(g(f(x)))$.\\ 
    Together they yield the desired equality $t(x) = v(x)$.
\end{proof}

\section{Subgroups}

\begin{definition}[Subgroup]
    Given a \textit{group} $(G, \ast)$, then the subset $H \subseteq G$ is a \textit{subgroup} of $G$ if it fulfills the properties:
    \begin{enumerate}
        \item \textbf{Closure}
        \begin{equation}
            \Forall x, y \colon x \in H \land y \in H \to x \ast y \in H
        \end{equation}
        \item \textbf{Neutral Element}
        \begin{equation}
            \epsilon \in H
        \end{equation}
        That is, the \textit{neutral} element $\epsilon$ from $G$ is contained within the subset $H \subseteq G$.
        \item \textbf{Invertibility}
        \begin{equation}
            \Forall x \in H \colon \Inverse{x} \in H
        \end{equation}
    \end{enumerate}
\end{definition}

\section{Lagrange's Theorem}

\begin{theorem}[Lagrange's Theorem]
    Given a finite \textit{group} of order $n$ $(G, \ast)$ where
    \begin{equation}
        G \coloneqq \set{ g_1, g_2, \dots, g_n }
    \end{equation}
    And its \textit{subgroup} $(H, \ast)$ of order $k \le n$
    \begin{equation}
        H \coloneqq \set{ h1_, h_2, \dots, h_k }
    \end{equation}
    Then $k \vert n$ ($k$ divides $n$).
    
    $G$ can be \textit{partitioned} into $\ell$ disjoint subsets of the same size $k$ such that
    \begin{equation}
        n = k \ell
    \end{equation}
\end{theorem}

\begin{definition}[Left Coset]
    Given $(G, \ast)$ is a \textit{group}, $(H, \ast)$ is a \textit{subgroup} of $(G, \ast)$ and $g \in G$ then the \textit{left coset} $gH$ of $H$ in $G$ with respect to $g$ is defined as
    \begin{equation}
        gH \coloneqq \set{g \ast h \colon h \in H}
    \end{equation}
\end{definition}

\begin{remark}
    Visually,
    \begin{equation}
        G \equiv \left. \begin{matrix}
            &\boxed{{g_1} H} \\
            &\boxed{{g_2} H} \\
            &\vdots \\
            &\boxed{{g_\ell} H}
        \end{matrix} \quad \right\} \ell \text{ disjoint subsets}
    \end{equation}
    
    To verify that the \textit{left cosets} together do in fact reconstruct $G$, check the multiplication table
    \begin{table}[H]
    \centering
    \begin{tabular}{c | c c c c}
    \toprule
    $\ast$     & $h_1$             & $h_2$             & $\cdots$ & $h_k$ \\ 
    \midrule
    $g_1 H$    & $g_1 \ast h_1$    & $g_1 \ast h_2$    & $\cdots$ & $g_1 \ast h_k$ \\
    $g_2 H$    & $g_2 \ast h_1$    & $g_2 \ast h_2$    & $\cdots$ & $g_2 \ast h_k$ \\
    $\vdots$   & $\vdots$          & $\vdots$          & $\ddots$ & $\vdots$ \\
    $g_\ell H$ & $g_\ell \ast h_1$ & $g_\ell \ast h_2$ & $\cdots$ & $g_\ell \ast h_k$ \\ 
    \bottomrule
    \end{tabular}
    \caption{Multiplication Table from $\ell$ Left Cosets, Each of Size $\lvert H \rvert = k$}
    \end{table}
\end{remark}

\begin{proposition}\label{prop:inverse-element-identity}
    For any $a, b \in G$ from $(G, \ast)$
    \begin{equation}
        \Inverse{(a \ast b)} \equiv \Inverse{b} \ast \Inverse{a}
    \end{equation}
\end{proposition}

\begin{proof}
    \begin{align}
        \Inverse{(a \ast b)} &\Leftrightarrow \Inverse{(a \ast b)} \ast \epsilon &\text{(Neutral element)} \\
        &\Leftrightarrow \Inverse{(a \ast b)} \ast (a \ast \Inverse{a}) &\text{(Invertibility)} \\
        &\Leftrightarrow \Inverse{(a \ast b)} \ast \left((a \ast \epsilon) \ast \Inverse{a}\right) &\text{(Neutral element)} \\
        &\Leftrightarrow \Inverse{(a \ast b)} \ast \left[(a \ast (b \ast \Inverse{b})) \ast \Inverse{a}\right] &\text{(Invertibility)} \\
        &\Leftrightarrow \Inverse{(a \ast b)} \ast \left[(a \ast b) \ast (\Inverse{b} \ast \Inverse{a})\right] &\text{(Associativity)} \\
        &\Leftrightarrow \left[ \Inverse{(a \ast b)} \ast (a \ast b) \right] \ast (\Inverse{b} \ast \Inverse{a}) &\text{(Associativity)} \\
        &\Leftrightarrow \epsilon \ast (\Inverse{b} \ast \Inverse{a}) &\text{(Invertibility)} \\
        &\Leftrightarrow \Inverse{b} \ast \Inverse{a} &\text{(Neutral Element)}
    \end{align}
\end{proof}

\begin{proof}
    For a constructive proof of Lagrange's Theorem: 
    
    Let the binary relation $E(x, y)$ be defined on the \textit{group} $(G, \ast)$, with its \textit{subgroup} $(H, \ast)$
    \begin{equation}
        E(x, y) \coloneqq \Inverse{x} \ast y \in H
    \end{equation}
    For the equivalence
    \begin{equation}
        x = y \Leftrightarrow \Inverse{x} \ast y = 1
    \end{equation}
    Then for each of the required properties:
    \begin{itemize}
        \item \textbf{Neutral Element} from \textit{Reflexivity} of $E(x, y)$
        \begin{equation}
            \Forall x \in G \colon E(x, x)
        \end{equation}
        Since
        \begin{equation}
            E(x, x) \equiv \Inverse{x} \ast x \in H \equiv \epsilon \in H
        \end{equation}
        Then this satisfies the \textit{reflexivity} requirement for  \textit{equivalence relations}, and proves the \textit{neutral element} requirement for \textit{subgroups}.
        \item \textbf{Invertibility} from \textit{Symmetry} of $E(x, y)$
        \begin{equation}
            \Forall x, y \in G \colon E(x, y) \to E(y, x)
        \end{equation}
        Let for some $h \in H$, $\Inverse{x} \ast y = h$, then by proposition \ref{prop:inverse-element-identity}
        \begin{equation}
            \Inverse{y} \ast x \equiv \Inverse{(\Inverse{x} \ast y)} \equiv \Inverse{h} \in H
        \end{equation}
        Which satisfies the \textit{symmetry} requirement for \textit{equivalence relations}, and proves the \textit{invertibility} requirement for \textit{subgroups}.
        \item \textbf{Closure} from \textit{Transitivity} of $E(x, y)$
        \begin{equation}
            \Forall x, y, z \in G \colon E(x, y) \land E(y, z) \to E(x, z)
        \end{equation}
        Let for some $h_1, h_2 \in H$, $\left( \Inverse{x} \ast y = h_1 \right) \land \left( \Inverse{y} \ast z = h_2 \right)$, then
        \begin{align}
            \Inverse{x} \ast z &\Leftrightarrow \Inverse{x} \ast \epsilon \ast z \\
            &\Leftrightarrow (\Inverse{x} \ast y) \ast (\Inverse{y} \ast z) \\
            &\Leftrightarrow h_1 \ast h_2 \in H
        \end{align}
        Which satisfies the \textit{transitivity} requirement for \textit{equivalence relations}, and proves the \textit{closure} requirement for \textit{subgroups}.
    \end{itemize}
    
\end{proof}

\begin{remark}
    To demonstrate Lagrange's Theorem, let the \textit{group} be constructed from $x \ast y \pmod{10}$.
    
    Let $(G, \ast)$ be a finite \textit{group} of order $n = 4$ where
    \begin{equation}
        G = \set{1, 3, 7, 9}
    \end{equation}
    And $(H, \ast)$ be its \textit{subgroup} of order $k = 2$.
    
    Constructing the multiplication table yields
    \begin{table}[H]
    \centering
    \begin{tabular}{c | c c}
    \toprule
    $\ast \pmod{10}$ & $1$ & $9$ \\
     \midrule
    $1 \ast H$       & $1$ & $9$ \\
    $3 \ast H$       & $3$ & $7$ \\
    $7 \ast H$       & $7$ & $3$ \\
    $9 \ast H$       & $9$ & $1$ \\ 
    \bottomrule
    \end{tabular}
    \caption{Multiplication Table for $(G, \ast)$}
    \end{table}
    
    There are only $\ell = 2$ disjoint subsets (unique cosets) $gH$; $G$ can be partitioned into $\ell$ disjoint subsets, each of size $\lvert H \rvert = 2$ such that $4 = n = k\ell = 2 \cdot 2$.
    
    Visually,
    \begin{equation}
        G = \left. \begin{matrix}
            &\boxed{1 \ast H = 9 \ast H = \set{1, 9}} \\
            &\boxed{3 \ast H = 7 \ast H = \set{3, 7}}
        \end{matrix} \quad \right\} \ell = 2
    \end{equation}
\end{remark}

\subsection{Equivalence Classes}

\begin{definition}[Equivalence Class]
    Given \textit{group} $(G, \ast)$ and its \textit{subgroup} $(H, \ast)$, then the \textit{equivalence class} $[g]$ is defined as
    \begin{equation}
        [g] \coloneqq \set{y \in G \mid \Inverse{g} \ast y \in H}
    \end{equation}
    Then
    \begin{equation}
        \Forall h \in H \colon \Inverse{g} \ast y = h \Leftrightarrow y = g \ast h
    \end{equation}
    Which yields the equivalence
    \begin{equation}
        \set{y \in G \mid \Inverse{g} \ast y \in H} \equiv \set{y \in G \mid y \in gH}
    \end{equation}
    Hence
    \begin{equation}
        [g] \equiv gH
    \end{equation}
    That the \textit{equivalence class} $[g]$ is exactly the \textit{left coset} $gH$.
    
    Let $\ell$ be the number of disjoint equivalence class $[g]$, then $G$ can be partitioned into $\ell$ disjoint subsets where visually,
    \begin{equation}
        G = \left. \begin{matrix}
            &\boxed{ [g_1] \equiv g_1 H } \\
            &\boxed{ [g_2] \equiv g_1 H } \\
            &\vdots \\
            &\boxed{ [g_\ell] \equiv g_\ell H } \\
        \end{matrix} \quad \right\} \ell \text{ disjoint subsets}
    \end{equation}
\end{definition}

\begin{proposition}
    \begin{equation}
        \Forall g \in G \colon \lvert gH \rvert \equiv \lvert H \rvert \equiv k
    \end{equation}
\end{proposition}

\begin{proof}
    Let $I$ be the set of indices $I \coloneqq \set{1, \dots, k}$
    \begin{align}
        &\Forall i, j \in I \colon (h_i = h_j) \leftrightarrow (g \ast h_i = g \ast h_j) \\
        \Leftrightarrow &\Forall i, j \in I \colon (h_1 \ne h_j) \leftrightarrow  (g \ast h_i \ne g \ast h_j)
    \end{align}
\end{proof}