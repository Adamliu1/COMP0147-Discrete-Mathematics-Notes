\chapter{Groups}

\section{Group Basics}

A \textit{group} is an abstract collection consisting of:
\begin{itemize}
    \item A \textit{nonempty set} $G$.
    \item A \textit{binary operation} $\star \colon G \times G \to G$.
\end{itemize}

It has the following properties:
\begin{enumerate}
    \item \textbf{Closure}
    \begin{equation}
        \Forall x, y \colon x \in G \land y \in G \to x \star y \in G
    \end{equation}
    \item \textbf{Associativity}
    \begin{equation}
        \Forall x, y, z \in G \colon (x \star y) \star z \equiv x \star (y \star z)
    \end{equation}
    \item \textbf{Neutral Element}
    \begin{equation}
        \Exists \epsilon \in G \colon \Forall x \in G \colon x \star \epsilon \equiv \epsilon \star x \equiv x
    \end{equation}
    That there exists an unique \textit{neutral} element $\epsilon \in G$.
    \item \textbf{Invertibility}
    \begin{equation}
        \Forall x \in G \colon \Exists y \in G \colon x \star y \equiv y \star x \equiv \epsilon
    \end{equation}
    That there exists an unique \textit{inverse} element $y \coloneqq \Inverse{x} \in G$ where $\Inverse{x}$ denotes the \textit{inverse} element of $x$.
\end{enumerate}

\begin{definition}[Commutative Group]
    An \textit{commutative group} (or \textit{abelian group}) is a \textit{group} for which its operation $\star \colon G \times G \to G$ satisfies the additional \textit{commutative} property:
    \begin{itemize}
        \item \textbf{Commutativity}
        \begin{equation}
            \Forall x, y \in G \colon x \star y \equiv y \star x
        \end{equation}
    \end{itemize}
\end{definition}

\section{Multiplicative Group}

\begin{proposition}[Multiplicative Group]
    A \textit{multiplicative group} is a \textit{group} $(G, \ast)$ which satisfies the required \textit{group} properties:
    \begin{itemize}
        \item \textbf{Closure, Associativity}. The multiplication operation $\ast \colon G \times G \to G$ is closed and is left associative.
        \item \textbf{Neutral Element}. The neutral element $\epsilon$ is unique.
        \item \textbf{Invertibility}. The inverse element $\Inverse{x}$ is unique.
        \item For all $a, b \in G$ the equation
        \begin{equation}
            a \ast x = b
        \end{equation}
        Has the unique solution
        \begin{equation}
            x = \Inverse{a} \ast b
        \end{equation}
    \end{itemize}
    Since
    \begin{align}
        a \ast x = b &\Leftrightarrow \Inverse{a} \ast (a \ast x) = \Inverse{a} \ast b &\text{(Multiply by $\Inverse{a}$ on both sides)} \\
        &\Leftrightarrow (\Inverse{a} \ast a) \ast x = \Inverse{a} \ast b &\text{(Associativity)} \\
        &\Leftrightarrow \epsilon \ast x = \Inverse{a} \ast b &\text{(Invertibility)} \\
        &\Leftrightarrow x = \Inverse{a} \ast b &\text{(Neutral Element)}
    \end{align}
\end{proposition}
