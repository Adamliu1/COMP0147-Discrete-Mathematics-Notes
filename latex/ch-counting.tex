\chapter{Counting}

\section{Counting Basics}

\subsection{[Multiplication Principle]}

\begin{definition}[Multiplication Principle]
    The \textit{multiplication principle} is used to count number of \textit{tuples} $(t_1, t_2, t_3, \dots)$ where $t_i$ are selected from \textit{independent} sources.
    
    For any sets $A_1, A_2, \dots, A_n$, their Cartesian product
    \begin{equation}
        \lvert A_1 \times A_2 \times \cdots \times A_n \rvert \equiv \VSBars{A_1} \cdot \VSBars{A_2} \cdot \cdots \cdot \VSBars{A_n}
    \end{equation}
\end{definition}

\begin{remark}
    For the set $E_2 = \set{0, 1}$,
    \begin{equation}
        \VSBars{{E_2}^3} = \VSBars{E_2 \times E_2 \times E_2} = 2^3 = 8
    \end{equation}
\end{remark}

\begin{remark}
    The number of boolean $n$-tuples is $2^n$
    \begin{equation}
        \VSBars{E_2^n} = \VSBars{ \underbrace{E_2 \times E_2 \times \cdots \times E_2}_{n} } = 2^n
    \end{equation}
\end{remark}

\begin{proof}
    For the Cartesian product $A \times B$ between any sets $A$ and $B$,
    \begin{equation}
        \VSBars{A \times B} \equiv \VSBars{A} \cdot \VSBars{B}
    \end{equation}
    \begin{table}[H]
    \centering
    \begin{tabular}{@{}c | cccc@{}}
    \toprule
             & $a_1$        & $a_2$        & $\cdots$ & $a_n$        \\ 
    \midrule
    $b_1$    & $(a_1, b_1)$ & $(a_2, b_1)$ & $\cdots$ & $(a_n, b_1)$ \\
    $b_2$    & $(a_1, b_2)$ & $(a_2, b_2)$ & $\cdots$ & $(a_n, b_2)$ \\
    $\vdots$ & $\vdots$     & $\vdots$     & $\ddots$ & $\vdots$     \\
    $b_k$    & $(a_1, b_k)$ & $(a_2, b_k)$ & $\cdots$ & $(a_n, b_k)$ \\ 
    \bottomrule
    \end{tabular}
    \end{table}
\end{proof}

\subsection{Addition Principle}

\begin{definition}[Addition Principle (Inclusion-Exclusion Principle)]
    For any sets $A$ and $B$,
    \begin{equation}
        \VSBars{A \cup B} \equiv \VSBars{A} + \VSBars{B} - \VSBars{A \cap B}
    \end{equation}
\end{definition}

\begin{remark}
    This is used in probability where for any events $A$ and $B$
    \begin{equation}
        \Prob{A \lor B} \equiv \Prob{A} + \Prob{B} - \Prob{A \land B}
    \end{equation}
\end{remark}
